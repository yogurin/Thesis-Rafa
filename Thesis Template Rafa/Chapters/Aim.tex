% Chapter 1

\chapter{Aim of the Thesis and Experimental Strategy} % Main chapter title

\label{Aim} 

% \lhead{Chapter 1. \emph{Introduction}} % This is for the header on each page - perhaps a shortened title

\graphicspath{{./Pictures/}}

%----------------------------------------------------------------------------------------

\section{Goals}

Viral egress affects the transmission and proliferation of virus progeny trough the host's tissue. For enveloped viruses, the late maturation steps and final egress via budding through the plasma membrane are well characterized. However, the current knowledge about late maturation, nuclear export, and egress of non-enveloped viruses remains largely unknown. 

The present thesis aims for a better understanding of the critical maturation steps leading to nuclear export and egress of a non-enveloped virus. The following issues are the main subject of this study: 




\medskip

\begin{enumerate}
\item Confirmation of the existence of an active process of nuclear export and egress of virions prior to passive release through cell lysis. 
\item Characterization of the critical capsid maturation steps that trigger active prelytic egress.
\end{enumerate}

\bigskip
\bigskip

\section{Experimental Strategy}

Three main experimental procedures were used:

\medskip

\begin{enumerate}
\item Fast protein liquid chromatography (FPLC) was used in order to separate, concentrate, and purify different intracellular virus populations representing distinct maturation intermediates of the parvovirus life cycle. Minute virus of mice (MVM) served as a model parvovirus to study late maturation steps, nuclear export, and egress of non-enveloped viruses. All experiments were performed using a restrictive murine cell line and/or a transformed human cell line.   
\item Standard biochemical and molecular biological methods were performed to investigate the structural and functional characteristics of the isolated virus populations.   
\item Site-directed mutagenesis on an infectious clone of MVM was applied to study the role of different capsid regions in active egress.
\end{enumerate}

