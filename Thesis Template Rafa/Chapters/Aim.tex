% Chapter 1

\chapter{Aim of the thesis and experimental strategy} % Main chapter title

\label{Aim} % For referencing the chapter elsewhere, use \ref{Chapter1} 

% \lhead{Chapter 1. \emph{Introduction}} % This is for the header on each page - perhaps a shortened title

\graphicspath{{./Pictures/}}

%----------------------------------------------------------------------------------------

\section{Goals}
The present thesis aims for a better understanding of the nuclear maturation steps which are involved in nuclear export and egress of parvovirus MVM. Parvovirus egress has direct implications in cell to cell spread and thus it affects the proliferation of virus progeny trough the host's tissue. In order to characterize the ultimate steps in parvovirus infection, the following basic approaches were elucidated in more detail:         

\medskip

\begin{enumerate}
\item Determination whether the egress of MVM is an active or passive process at the end of an infection. 
\item Identification of the critical maturation steps that trigger nuclear export and pre-lytic egress in the case of an active egress of MVM progeny.  
\end{enumerate}

\bigskip
\bigskip

\section{Experimental strategy}

To answer the aforementioned goals, three main experimental approaches were taken:

\medskip

\begin{enumerate}
\item Fast protein liquid chromatography (FPLC) was used in order to separate, concentrate, and purify different intracellular virus populations representing distinct maturation intermediates of the parvovirus life cycle. 
\item Standard biochemical and molecular biological methods were performed to investigate the structural and functional characteristics of the isolated virus populations.   
\item Site-directed mutagenesis on an infectious clone of MVM was applied to challenge obtained results and reinforce concluding observations.
\end{enumerate}

