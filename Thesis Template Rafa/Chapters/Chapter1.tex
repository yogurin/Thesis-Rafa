% Chapter 1

\chapter{Discovery and brief history} % Main chapter title

\label{Chapter1} % For referencing the chapter elsewhere, use \ref{Chapter1} 

% \lhead{Chapter 1. \emph{Introduction}} % This is for the header on each page - perhaps a shortened title

\graphicspath{{./Pictures/}}

%----------------------------------------------------------------------------------------
\label{sec:Discovery and brief history}
Minute virus of mice (MVM) is a small, non-enveloped autonomous replicating parvovirus. Two variant forms of MVM that share 96~\% nucleotide (nt) sequence identity \cite{pmid3855242} have been discovered. 
MVMp, the prototype strain, was isolated and characterized by Crawford in 1966. It originated from a contaminated murine adenovirus stock and was shown to replicate efficiently in mouse fibroblasts \cite{pmid5945715}. The virus was plaque purified in 1972 \cite{pmid4673484} and the resulting strain was designated MVM(p) for prototype \cite{MVMp}. Another strain was recovered from the culture fluid of infected murine EL-4 T-cell lymphoma cells by Bonnard and colleagues in 1976 \cite{pmid1244418}. This strain efficiently replicates in lymphocytes and is immunosuppressive for allogeneic mixed leukocyte cultures as it inhibits the generation of cytolytic T lymphocytes \cite{pmid6457871}. Therefore, it was referred to as immunosuppressive strain MVMi \cite{pmid6264106}. Both strains are well characterized and reciprocally restricted for growth in each other’s murine host cell.  

Since its discovery nearly 50 years ago, MVM served as an interesting model virus to dissect the molecular mechanisms of tissue tropism, capsid dynamics associated with endosomal trafficking, as well as viral deoxyribonucleic acid (DNA) replication and packaging. Furthermore, it gained increasing interest as an important tool for cancer therapy due to its oncolytic capabilities and currently represents a commonly accepted parvovirus model.     

