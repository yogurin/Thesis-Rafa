

\chapter*{Abstract}
\addcontentsline{toc}{part}{Abstract}
% Main chapter title


\label{Abstract} % For referencing the chapter elsewhere, use \ref{Chapter1}


The active egress of enveloped viruses is a well-characterized process and involves budding through host cell membranes. The release of non-enveloped viruses is considered a passive process because it is associated with cellular lysis. However, this basic principle in virology has recently been challenged by several studies suggesting that non-enveloped viruses may also egress by an active process but the mechanisms involved remain largely unknown. Due to their simplicity, the non-enveloped parvoviruses are strongly dependent on host cell functions for their replication and proliferation. Therefore, they are an ideal tool to study virus host-cell interactions. In order to gain insights into the mechanisms involved in the egress of parvoviruses, the late virus maturation steps preceding virus release were studied in two different cell lines. 
	
\par
\medskip
Minute virus of mice (MVM) is a well-known model parvovirus. Assembly of structural capsid proteins occurs in the nucleus giving rise to icosahedral empty capsid (EC) precursors which are subsequently filled with the viral single-stranded DNA to generate full capsids (FC). By performing anion-exchange chromatography, intranuclear MVM progeny particles were separated based on their net surface charge. Apart from EC, two distinct FC progeny populations arose in the nuclei of infected cells. The first FC population to appear was fully infectious but was nuclear export deficient. In order to acquire egress potential, this early FC progeny underwent further maturations involving the exposure of the N-termini of the major capsid protein VP2 (N-VP2), as well as phosphorylations of surface residues. While the surface phosphorylations were strictly associated to nuclear export capacity, mutational analysis revealed that the phosphoserine-rich N-VP2 was dispensable. Mutants with modified N-VP2 showed less efficient delivery of the viral genome to the nucleus, revealing an important role of N-VP2 in assisting virus entry. The fact that only the mature phosphorylated population of FC was able to escape from the cells before detectable cell lysis confirms the existence of an active process of virus egress. For cell entry the reverse situation was observed since during endocytic trafficking, incoming virions lost both the N-VP2 termini and the additional surface phosphorylations. 
	
\par
\medskip
Collectively, these temporally and spatially controlled changes in capsid surface phosphorylation would provide nuclear import and export potential required to complete the life cycle of the karyophilic virus. Further studies are required to identify the corresponding phosphorylations on the capsid surface and to demonstrate their specific role in the active egress of the non-enveloped parvovirus. 