% Chapter 6

\chapter{Host Range and Specificity} % Main chapter title

\label{Chapter6} % For referencing the chapter elsewhere, use \ref{Chapter5} 

% \lhead{Chapter 2. \emph{Methods}} % This is for the header on each page - perhaps a shortened title

%----------------------------------------------------------------------------------------

\section{Tissue Tropism Determinants}

Concerning their host range, most parvoviruses, such as MVM, \nomenclature{MVM}{Minute virus of mice} CPV, \nomenclature{CPV}{Canine parvovirus} and FPV, \nomenclature{FPV}{Feline parvovirus} are tightly restricted to specific receptors of their particular hosts. However, some parvoviruses, as for example many of the AAVs, \nomenclature{AAV}{Adeno-associated virus} infect human cells by primary attachment to a variety of receptors (see Section~\ref{Binding}, p.~\pageref{Binding}). 

As outlined in Chapter~\ref{sec:Discovery and brief history} (see p.~\pageref{sec:Discovery and brief history}), two distinct strains of the parvovirus MVM have been described to occur in mice. On the one hand, MVMp, \nomenclature{MVMp}{Prototype strain of MVM} the prototype strain, replicates efficiently in mouse fibroblasts \cite{pmid5945715}. On the other hand, MVMi,\nomenclature{MVMi}{Immunosuppressive strain of MVM} the immunosuppressive strain, replicates in T lymphocytes \cite{pmid1244418, pmid6264106}.
Both strains display disparate \textit{in vitro} tropism and \textit{in vivo} pathogenicity despite differing by only 14 amino acids in their capsid proteins \cite{pmid1871965}, thus providing a useful model for in-depth characterization of the role of virus-receptor interaction (see Section~\ref{Binding}, p.~\pageref{Binding}) in parvovirus infection. Beyond that, MVMp and MVMi are serologically indistinguishable, bind to sialic acid (SA), and are internalized in both fibroblasts and lymphocytes \cite{pmid6602221}. Consequently, it could be demonstrated that both viruses propagate in hybrids of the two cell types \cite{pmid6602222}.    

In order to map the allotropic determinants of MVM, chimeric viral genomes were constructed \textit{in vitro} from infectious genomic clones of both strains. By mutagenesis and selective plaque assays, the major determinants for the acquisition of fibrotropism for MVMi have been mapped onto the capsid \cite{pmid3257270, pmid9519837, pmid1871965}, in particular to the VP2 residues 317 and 321 \cite{pmid7637028, pmid1316457}. Both residues are located at the base of the threefold spike of the virion \cite{pmid3357208, pmid3392768, pmid3257270}. Interestingly, these two VP2 residues structurally localize nearby some of the important amino acids determining CPV, FPV, and PPV host range \cite{pmid14581558, pmid12941920, pmid1532105}. Further residues (VP2 residues 339, 399, 460, 553, and 558) were identified in MVMi to be able to confer fibrotropism to forward second-site mutants when either residues 317 or 321 are mutated. Those residues cluster around the twofold dimple-like depression \cite{pmid9817841}. In contrast, the switch to lymphotropism for MVMp is more complex and requires both an equivalent region of the major MVMi capsid protein gene VP2 and a segment of the NS protein genes \cite{pmid9519837}.    


\section{Pathogenicity Determinants}

MVMi appears to be more pathogenic in mice than MVMp. Oronasal inoculation of MVMi in most neonatal mice resulted in lethal phenotype or severe growth-retardation in survivors \cite{pmid3712557}, as observed for other parvoviruses (see Section~\ref{sec: The Parvovirinae subfamily}, p.~\pageref{sec: The Parvovirinae subfamily}). MVMp infection appears to be asymptomatic in newborn mice \cite{pmid1373202}. In contrast, MVMi infection in neonatal mice of some inbred strains caused renal papillary hemorrhage and viral replication in endothelia \cite{pmid1653878}, hematopoietic precursors \cite{pmid7707557}, and neuroblasts \cite{pmid8892936}. Following \textit{in utero} inoculation of MVMi or MVMp into developing embryo, a broad set of cell types were infected that partially overlapped. Nevertheless, the tissue tropism of MVMp for fibroblasts and of MVMi for endothelium, as well as the higher virulence of MVMi was preserved \cite{pmid15308740}. 
By reason of the complexity of MVMi pathogenesis in the neonatal mouse, a more adequate model was required to investigate the virulence of MVMi \textit{in vivo}. 

Severe combined immunodeficiency (SCID) mice \cite{pmid6823332} represent such a model since they lack an antigen-specific immune response, thus allowing the study in adult mice and circumventing the complex situation of heterogenous viral multiplication in embryonic developing tissue. MVMi infection of adult SCID mice gave rise to the suppression of long-term repopulating hemopoietic stem cells in the bone marrow \cite{pmid12857918}, leading to an acute lethal leukopenia and accelerated erythropoiesis \cite{pmid9971754}.
In addition, it has been reported that MVMp evolved in intravenously inoculated SCID mice. Different variants, isolated from single plaques, carried only one of three single amino acid changes at position 325, 362, or 368 in the major VP2 capsid protein. These variants sustained their fibrotropism \textit{in vitro}, but unlike MVMp, they propagated in mouse tissues following oronasal inoculation, eventually causing death \cite{pmid16415031, pmid16103180}.
Two of the three invasive fibrotropic MVMp strains, I362S and I368R, were shown to induce lethal leukopenia in oronasal inoculated SCID mice. Emerging viral populations in leukopenic mice displayed altered sequences in the MVMi genotype at position 321 and 551 of VP2 for infections with the I362S variant or changes at position 551 and 575 in the K368R virus infections. In general, a high level of genomic heterogeneity in the DNA sequence encoding the VP2 protein was observed and was found to be clustered at the twofold depression of the viral capsid \cite{pmid18045943}.            	
	
\clearpage	
\section{Comparison of Tissue Tropism and Pathogenicity Determinants among Parvoviurses}	
	
	Significantly, the amino acids dictating \textit{in vitro} tropism (317 and 321), \textit{in vivo} pathogenicity (325, 362, and 368), fibrotropism on MVMi (339, 399, 460, 553, and 558), and those involved in the development of leukopenia (321, 551, and 575) were found to be located on, or near the capsid surface. Structurally, these residues cluster mainly by raised elements around the twofold axes of symmetry, in close vicinity of the SA binding pocket (see Section~\ref{Binding}, p.~\pageref{Binding}) \cite{pmid16415031, pmid18045943}.   
	
Differences in the tissue tropisms and the pathogenic phenotypes  have also been mapped to the capsid proteins of Aleutian mink disease parvovirus \cite{pmid8396664}, PPV \nomenclature{PPV}{Porcine parvovirus} \cite{pmid8642680}, CPV \cite{pmid3176341, pmid1331498}, and FPV \cite{pmid7513918} in a capsid region analogous to that observed for MVM (reviewed in \cite{tropism}). These pronounced \textit{in vitro} tropism and \textit{in vivo} pathogenicity disparities between the highly homologous viruses can occur at any of the various stages of the infectious viral life cycle, including cell receptor binding, internalization, capsid uncoating, DNA replication or transcription. Studies of the strain-specific tissue tropism conducted on members of other virus families have mainly shown that each strain recognizes a different specific cell surface receptor \cite{pmid6290894, pmid13908368, pmid6293181, pmid6278730, pmid7436739, pmid7365250, pmid271999}. This receptor is only present on the target cell for that strain, but absent on the surface of other potential host cells. Although the same structural elements of parvoviruses are involved in mediating host and tissue tropisms, each appears to be affecting a different mechanism. Host ranges of CPV and FPV are controlled by receptor binding, as observed for many other viruses, whereas the cell tropisms of MVM appear to be due to restrictions of interactions with intracellular factors \cite{pmid9817841, pmid12941411, pmid6602221}. For MVM it was suggested that the point of restriction appeared after nuclear targeting and conversion of genomic ssDNA to replicative form (RF) intermediates but prior to viral genome replication. Most likely, the restraint occurs due to a block in capsid uncoating \cite{pmid9311862, pmid1322591}.

As discussed in this section the functional regions among the subfamily \textit{Parvovirinae} co-localize to similar capsid surface regions albeit three general parvovirus topology groups with characteristic local morphological surface differences emerged (see Chapter~\ref{sec:Morphology}, p.~\pageref{sec:Morphology}). A profound understanding of functional domains that are involved in fundamental steps of the viral life cycle, particularly receptor attachment, \textit{in vitro} tropism, \textit{in vivo} pathogenicity, and antigenicity are essential for infection and disease control. Hence, showing great promise to allow genetic engineering of parvovirus capsids for the therapeutic delivery to be controlled or modified in gene therapy applications and to develop foreign antigens \cite{pmid12941411, tropism}.
\label{Chaper6end}     



%Although MVMi infection may result in pathology of infected mice, it has been shown that the infection more likely interferes with numerous T-cell functions \textit{in vitro}. The infection rather causes problems for the ongoing study the mice are being used for as the immune system will be activated, the activity of T-lymphocytes or B-lymphocytes will be altered and tumor formation may be suppressed \cite{pmid6457871, pmid6264106, pmid11528091}.

\nomenclature{SCID}{Severe combined immunodeficiency}
\nomenclature{RF}{Replicative form}