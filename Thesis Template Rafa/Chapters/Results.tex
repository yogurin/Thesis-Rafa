% Chapter XY

\chapter{Results} % Main chapter title

\label{Results} % For referencing the chapter elsewhere, use \ref{Chapter2} 

% \lhead{Chapter 2. \emph{Methods}} % This is for the header on each page - perhaps a shortened title

%----------------------------------------------------------------------------------------

\section{Comparison of ECs and FCs in early virus infection}

In addition to the previously characterized FC populations (FC-P\textsubscript{1} and FC-P\textsubscript{2}), infected cells produce also a considerable amount of ECs. Due to the lack of DNA, EC band at lower density in a CsCl gradient compared to FC. While FC entered the gradient to a density of 1.46 gcm\textsuperscript{-3}, EC already banded at 1.32 gcm\textsuperscript{-3}, as determined by refractometry. A quantitative PCR analysis of the corresponding fractions confirmed that viral DNA containing particles were depleted from ECs to almost a thousand times. Approximately half of the overall viral progeny population represent ECs. Therefore, it is of interest to characterize their role during the course of infection. We studied their ability to bind to restrictive murine cells, their capacity to compete with FCs, and their potential to interfere with the progression of a natural infection.    

\subsection{Both FC and EC bind specifically to SA residues on the cell surface}
In order to characterize the binding specificity of FC and EC, both capsid types were allowed to bind discretely to susceptible, restrictive murine fibroblasts at 4~\textcelsius. At such low temperature, active cell-mediated uptake through endocytosis is prohibited. Unbound viruses were removed by several washings of the adherent cells. For restrictive mouse fibroblasts, binding saturation is reached at MOIs higher than \np{10000} DNA-containing particles per cell, as determined by the quantification of bound FCs to adherent cells at 4~\textcelsius. Both capsid species restrictively bind to SA residues since they can be completely depleted from the cell surface by treatment with neuraminidase (see Table~\ref{Enzymes}, p.~\pageref{Enzymes}), an enzyme that specifically hydrolyzes glycosidic linkages of neuraminic acids. Complete removal of attached viruses is even achieved under saturated conditions. In order to guarantee a complete removal of viral particles from the cellular surface, a minimal dose of 25 U/mL of the enzyme is required.   

\subsection{EC and FC exhibit morphological differences}
The flexible, unordered N-terminus of the major structural protein, VP2, shows distinct conformation in either capsid population. N-VP2 is accessible to proteolytic digestion or specific antibodies only in FC, whereas it remains inaccessible in EC, indicating important structural differences between these capsid populations. The differences for N-VP2 accessibility can be used to distinguish FCs and ECs in IF experiments. Staining of FCs results in co-localization of $\alpha$-Caps and $\alpha$-N-VP2 antibodies whereas ECs are detected only by $\alpha$-Caps antibodies.   

    


\subsection{FC are preferentially bound to the SA residues on the cell surface}
In silico quantification of co-localization in representative IF pictures revealed that binding of FCs was not disturbed in the presence of ECs. When FCs and ECs were bound to cells at equal stoichiometry, FCs preferentially bound to the cell surface, indicating a higher binding affinity for FCs compared to ECs. Co-localization of both antibodies was higher than 95 \% in the absence and in the presence of ECs, indicating that FCs bound preferentially to the cells. Even under non-saturated conditions, ECs were detected rarely when applied as mixed populations. Only when an equal amount of ECs was added prior to the FCs a slight increase in bound ECs was observed. Nevertheless, ECs did not represent 50 \% of the bound population but only reduced co-localization marginally to approximately 75 \%.           

\subsection{EC do not compete with FC for cell surface receptors}
Quantitative competition experiments under saturated conditions confirmed that increasing amounts of ECs did not disturb the attachment of FCs to the cell surface. These results substantiate the preferential binding of FCs to susceptible cells previously observed in IF experiments. Even an unnatural 16-fold excess of ECs did not significantly disturb receptor binding of FCs. 


Due to the differences in N-VP2 conformation among FCs and ECs, there is evidence that the N-VP2 termini may be involved in the stabilization of the binding   

