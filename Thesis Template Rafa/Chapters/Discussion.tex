% Chapter 

\chapter{Discussion} % Main chapter title

\label{Discussion} % For referencing the chapter elsewhere, use \ref{Chapter1} 

% \lhead{Chapter 1. \emph{Introduction}} % This is for the header on each page - perhaps a shortened title

\graphicspath{{./Pictures/}}

%----------------------------------------------------------------------------------------

Recent research in our lab focused on dynamics and structural rearrangements on the surface of parvovirus capsids during the first stages of an infection. The studies were conducted on both parvoviruses B19V and MVM. For the latter model virus, three major pH-dependent structural capsid rearrangements were observed \textit{in vivo}. These include the proteolytic digestion of N-VP2, the externalization of N-VP1 through the cylindrical 5-fold channel, and the externalization of viral DNA which remained associated to the capsid (see Section~\ref{Rearrangements}, p.~\pageref{Rearrangements}) \cite{pmid16379002}. To date, neither the triggers causing these endosomal rearrangements nor their purpose have been elucidated in detail. B19V has been shown to undergo conformational changes upon binding to its primary attachment receptor globoside (Gb4Cer), also referred to as erythrocyte P antigen. Binding to the receptor triggers the externalization of VP1u, the highly immunogenic N-terminus of the minor structural capsid protein VP1 \cite{pmid20826697}. VP1u has been shown to be the key determinant for the extraordinarily restricted B19V tropism \cite{pmid24067971}. Most recent results obtained in our lab indicate that B19V undergoes phosphorylation events during cytoplasmic trafficking leading to significant structural rearrangements of the B19V capsid and eventual DNA externalization. While incoming cytoplasmic capsids lost the conformational epitopes targeted by an antibody against assembled capsids, antibodies against phosphorylated serine residues efficiently recognized partially disassembled intermediates. The capsid-tethered genomic DNA of these partially uncoated particles was accessible for \textit{in vitro} extension with specific primers complementary to both ITRs. (Ruprecht, N. \textit{et al.}, manuscript in preparation).  

As previously mentioned, parvoviruses are highly robust in the extracellular milieu resisting harsh physicochemical conditions (see Section~\ref{Physicoprop}, p.~\pageref{Physicoprop}). However, \textit{in vivo} they need to uncoat their capsid shell in order to ensure an efficient transmission of their genomic DNA into the host’s nucleus. These structural rearrangements are triggered by numerous molecular interactions between the host cell and the virus (see Chapter~\ref{Chapter7}, pp.~\pageref{Cycle}~-~\pageref{Egress1}). In particular, the viral surface being comprised of extended and highly dynamic loop structures (see Section~\ref{Structure}, p.~\pageref{Structure} and Figure~\ref{Structure1}, p.~\pageref{Structure1}), is exposed to cellular receptors and enzymes. Cell-mediated modifications, such as phosphorylation, proteolytic digestion, or ubiquitination, trigger structural rearrangements and ultimately lead to the disassembly of the virus particle. Consequentially, these host cell-mediated capsid surface modifications alter the net surface charge of viral particles. Therefore, viral populations representing a distinct maturation stage might display surface features which are different from the native status.   

FPLC is a high performance chromatography method that combines several advantages. First, the small-diameter stationary phase enables high resolution and fast flow rates. Secondly, samples can be diluted in biocompatible aqueous buffer systems and large sample volumes can be injected to the system. Thirdly, separation is highly reproducible due to a high level of automation including gradient program control and fraction collection. Finally, a full range of chromatography modes, such as ion exchange, chromatofocusing, gel filtration, hydrophobic interaction, and reverse phase can be provided \cite{pmid20978981}. 








\nomenclature{Gb4Cer}{Globotetraosyl ceramide}