

\chapter*{Zusammenfassung}
\addcontentsline{toc}{part}{Zusammenfassung}
% Main chapter title


\label{Zusammenfassung} % For referencing the chapter elsewhere, use \ref{Chapter1}

Behüllte Viren werden von ihrer Wirtszelle sezerniert, indem sie bei der Knospung Stücke von deren Zellmembran als Bestandteil in die Virushülle integrieren. Für behüllte Viren ist die aktive Freisetzung von Virionen aus infizierten Wirtszellen bereits gut dokumentiert. Die Freisetzung von unbehüllten Viren wird als passiver Vorgang beschrieben, bei dem die Zellmembran durch Zelllyse aufgelöst wird, was zur Absonderung der neu gebildeten Virionen führt. Dieser Grundsatz der Virologie wurde in jüngster Zeit durch mehrere Studien in Frage gestellt, die für unbehüllte Viren einen aktiven Mechanismus zur Freisetzung der Virionen aufzeigten. Aufgrund ihrer Einfachheit sind die unbehüllten Parvoviren zur eigenen Vermehrung und Verbreitung stark von ihrer Wirtszelle abhängig. Daher eignen sie sich gut um die Interaktionen zwischen Viren und ihren Wirtszellen zu studieren. Um die Freisetzung von Parvoviren aus ihrer Wirtszelle besser zu verstehen, wurden späte Maturationsschritte im Kern von infizierten Zellen untersucht, welche  der Absonderung der neu gebildeten Virionen unmittelbar bevorstehen. Zur Verstärkung der Aussagekraft wurden die Experimente an zwei verschiedenen Zelllinien durchgeführt. 

\par
\medskip
Minute virus of mice (MVM) ist ein gut charakterisiertes Parvovirus, das sich als Modell für diese Studie eignet. Das Kapsid von Parvoviren wird aus Strukturproteinen, den sogenannten Kapsomeren, gebildet. Beim Vorgang der Selbstassemblierung lagern sich im Zellkern 60 solcher Kapsomere spontan und ohne Energieverbrauch zusammen und bilden ein Kapsid mit ikosaedrischer Symmetrie. Diese leeren Kapside werden anschliessend mit der einzelsträngigen viralen DNA bepackt. Neu generierte Virus Partikel wurden in dieser Studie mittels A\-ni\-o\-nen\-aus\-tausch Chromatographie nach deren Oberflächenladungen aufgetrennt. Neben den leeren Kapsiden wurden auf diese Weise zusätzlich zwei unterschiedliche Populationen DNA-bepackte virale Partikel separiert. Die erste Population bepackter Viren im Zellkern war infektiös, wurde allerdings nicht ins Zytoplasma exportiert. Der Export aus dem Zellkern wurde der zweiten Population durch weitere Maturation ermöglicht. Die Maturationsschritte beinhalteten die Externalisierung der Amino-Termini des Hauptstrukturproteins VP2 (N-VP2) durch die Poren an den 5-fach Symmetrieachsen des Kapsids, sowie Phosphorylierungen von Aminosäuren auf der Oberfläche der Kapside. Die Phosphorylierungen auf der Kapsidoberfläche wurden jeweils nur bei den Kapsiden beobachtet, die aus dem Kern exportiert wurden. Mutanten mit modifizierten N-VP2 Termini konnten hingegen ungehindert aus dem Zellkern exportiert werden. Dies impliziert, dass N-VP2 für diesen Vorgang entbehrlich ist. Es zeigte sich, dass N-VP2 vor allem zur Initiierung der viralen Infektion, sowie zur Reorganisation des Zytoskeletts während der Freisetzung der Viren erforderlich war. Die N-VP2 Mutanten wiesen einen erschwerten Transport des Kapsids zum Zellkern auf. Im Kern wurde weniger mutierte virale DNA quantifiziert was eine verzögerte Replikation zur Folge hatte. Zudem war die Zytolyse infizierter Zellen deutlich verzögert.  

\par
\medskip
Weil vor der Zelllyse ausschliesslich der vollständig maturierte Virus im Überstand der Zellkultur nachgewiesen wurde, konnte ein aktiver Mechanismus zur Freisetzung der neu gebildeten Virionen bestätigt werden. Obwohl zuerst eine Segregation beider DNA-bepackten Populationen beobachtet wurde konnte die frühe, unreife virale Population durch virus-induzierte Zellyse passiv freigesetzt werden. Interessanterweise wurde beim Eindringen der Viren in die Wirtszelle die umgekehrte Situation beobachtet. In den Endosomen wurde N-VP2 proteolytisch abgebaut und saure Phosphatasen entfernten die Phosphorylierungen auf der Kapsidoberfläche. 

\par
\medskip
In dieser Arbeit wurden Phosphorylierungen auf der Kapsidoberfläche eines unbehüllten Virus identifiziert welche zeitlich und räumlich streng kontrolliert werden. Diese Modifikationen auf der Kapsidoberfläche könnten unbehüllten Viren den Import in den Zellkern, beziehungsweise deren Export aus dem Zellkern, ermöglichen. Der Transport in und aus dem Zellkern ist für karyophile Viren unabdingbar zur Vollendung ihres Lebenszyklus. Zur Identifizierung der entsprechenden Phosphorylierungen auf der Kapsidoberfläche sind in Zukunft weitere Untersuchungen notwendig. Zudem bleibt deren spezifischer Einfluss in der aktiven Freisetzung der neu generierten Virionen von der Wirtszelle weiterhin unklar.
