% Chapter 2

\chapter{Methods} % Main chapter title

\label{Chapter2} % For referencing the chapter elsewhere, use \ref{Chapter2} 

\lhead{Chapter 2. \emph{Methods}} % This is for the header on each page - perhaps a shortened title

%----------------------------------------------------------------------------------------

\section{Cell Cultures}
A9 ouab\textsuperscript{r}l1 cells, a derivative from the original HGPRT\textsuperscript{-} L-cell line A9 represent a clone resistant to 10\textsuperscript{-3} M ouabain after nitrosoguanidine mutagenesis \cite{pmid14213660}.
NB324K cells are a clone of SV40-transformed \nomenclature{SV40}{Simian vacuolating virus 40 or Simian virus 40} human newborn kidney cells \cite{pmid13911591}. The SV40 large T antigen was detected by immunofluorescent \nomenclature{IF}{Immunofluorescence} staining with monoclonal antibodies \cite{pmid6169844}. \nomenclature{mAb}{Monoclonal antibody} However, NB324K cells do not produce infectious SV40 spontaneously.
Both cell lines, A9 mouse fibroblasts and NB324K cells, were routinely propagated under a minimal number of passages in DMEM \nomenclature{DMEM}{Dulbecco modified Eagle's medium}supplemented with 5 \% of heat inactivated fetal bovine serum at 37 \textcelsius~ in 5 \% CO\textsubscript{2} atmosphere. \nomenclature{FCS}{Fetal calf serum} 

\subsection{Freezing and thawing of cells}
Before use the A9 mouse fibroblasts or NB324K cells were thawed at 37 \textcelsius~ and cultured in 5 mL of pre-warmed DMEM supplemented with 5 \% FCS. The medium was replaced every 3 to 4 days. 
In order to freeze the cells for long storage in liquid nitrogen they were passed the day before, to ensure exponential growth. Subsequently, 7.5 \% DMSO was added and the cells were frozen slowly at -70 \textcelsius~ over night before transfer to liquid nitrogen.

%----------------------------------------------------------------------------------------

\section{Virus Stocks}
\label{Virus Stocks}
Stocks of MVM without detectable levels of VP3 were propagated on A9 mouse fibroblast monolayers. As soon as the cytopathic effect became evident, the supernatant \nomenclature{SN}{Supernatant}was collected and pre-cleared from cell debris by low-speed centrifugation. Thereby, intracellular, VP3 rich capsids were discarded. In order to remove low-molecular contaminants, virus containing SN was pelleted through 20 \% sucrose cushion in PBS by ultra-centrifugation. Virus titers were determined by qPCR \nomenclature{qPCR}{Quantitative PCR} \nomenclature{PCR}{Polymerase chain reaction} as DNA-packaged particles per microliter.   

\subsection{Separation of empty and full capsids}

Sucrose purified capsids were prepared as previously described in section \ref{Virus Stocks}, page \pageref{Virus Stocks}. The virus pellet was resuspended in 10 mL PBS. Caesium chloride was added to a density of 1.38 g/mL adjusted by refractometry ($\eta$=1.371) at 4 \textcelsius. The gradient was centrifuged to equilibrium for 24 h at 41000 rpm and 4 \textcelsius~ in a Beckmann SW-41 Ti rotor. Gradients were fractionated and tested for intact capsids by dot blot analysis using B7 mAb. CsCl was depleted from the corresponding fractions by size-exclusion chromatography through PD-10 desalting columns and concentrated by ultra-centrifugation when required.          
   
   








\section{Freezing bacteria stocks in glycerol}
Bacteria were frozen in dry ice. A volume of 700 $\mu$L of the bacteria culture that was grown over night in LB-medium was mixed with 300 $\mu$L of 50 \% glycerol in a cryotube. In order to mix well the glycerol the cryotube was vortexed intensively. Following snap-freeze in dry ice the bacteria were stored at -70 \textcelsius.



\section{Anion-exchange chromatography}
A Mono Q HR 5/5 (Pharmacia) column (5 x 50 mm) was used to analyse viral samples. The Mono Q column was connected to the ÄKTAmicro chromatography system (GE Healthcare) that was operated by the UNICORN control software. The Mono Q column was equilibrated with six column volumes (CV) starting buffer (20 mM Tris-HCl, 1 mM EDTA, pH 7.2). Samples (1 mL) containing at least 10\textsuperscript{10} virus particles in 10 mM Tris-HCl, 1 mM EDTA, pH 8 were applied to the Mono Q column trough a 2 mL loop. After eluting the protein, which did not bind to the column in the starting buffer, a linear salt gradient (0-2 M NaCl) in 20 mM Trsi-HCl, 1 mM EDTA, pH 7.2, was applied. Fractions of 0.185 mL were collected in 96-well plates. Viral genomes in each fraction were quantified by qPCR. All buffers were filtered and degassed before application to the Mono Q column. 





\subsection{}


\subsection{}

\subsection{}
 

%----------------------------------------------------------------------------------------

\section{}


\subsection{}



\subsection{}

